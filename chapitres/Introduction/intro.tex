Contexte du Projet:
\newline
-L'Ob\'esit\'e:
\begin{itemize}
\item{Qu'est ce que l'ob\'esit\'e?}
\item{Pourquoi l'obésité est une maladie dangereuse?}
\item{Est ce que l'ob\'esit\'e est une maladie transmissible?}
\item{Quelle est la relation entre l'ob\'esit\'e et les r\'eseaux?}
\end{itemize}
[L'ob\'esit\'e: d\'efinition]
En 1997, l'ob\'esit\'e a \'et\'e reconnue comme une maladie par l'OMS (Organisation Mondiale de la sant\'e)
[Risques de l'ob\'esit\'e]
[Caractère transmissible do l'ob\'esit\'e]
La propagation des maladies transmissibles est l'un des probl\`emes qui n'a pas encore de solution concrète, peut-\^etre on n'a pas les moyens qui nous permettent d'emp\^echer/.. la propagation de certains ph\'enom\`enes, cependant, dans une certaine mesure, on peut les contr\^oler pour diminuer les d\'eg\^ats. G\'en\'eralement le ph\'enom\`ene de propagation n'est pas limit\'e/r\'eserv\'e/li\'e seulement aux maladies, il y en a d'autres domaines/cas/situations qui v\'erifient/poss\`edent la propri\'et\'e des maladies transmissibles; les m\'edias sociaux illustrent bien le ph\'enom\`ene de propagation, cette fois-ci ce sont les informations, le buzz et m\^eme les \'emotions qui se propagent/partagent. Pour \'etudier un tel probl\`eme, les graphes peuvent \^etre une bonne m\'ethode de mod\'elisation et par cons\'equence les r\'eseaux aussi dans le cas d'un probl\`eme dynamique qui varie au cours du temps.
\newline
Probl\'ematique et Objectifs:
\newline
Ce projet de fin d'\'etudes a comme objectif la cr\'eation/recherche/ d'une plate-forme qui permet l'analyse et la visualisation des r\'eseaux afin d'en extraire les informations n\'ecessaires pour l'\'etude d'un probl\`eme donn\'e. Souvent les r\'eseaux sont repr\'esent\'es sous forme de tableaux ou matrices, cette pr\'esentation est favorable pour la manipulation des \'ees, mais d'un point de vue visuel, il est difficile de suivre l'\'evolution/variations/changements dans un r\'eseau en se basant seulement sur la lecture des donn\'ees \`a partir des tableaux. Donc une repr\'esentation graphique des tableaux peut bien r\'epondre au probl\`eme, les graphes sont de bons moyens  pour faire parler/expliquer/visualiser/ les r\'eseaux. Les outils d'analyse et de visualisations des graphes sont multiple, chaque outil a ses propres caract\'eristiques, donc le choix d\'epend de la nature du r\'eseaux ainsi que ces caract\'eristiques. 
